\documentclass[10pt]{amsart}
\usepackage[margin=1.5in]{geometry}
\usepackage{amssymb,amsmath,enumitem}

\DeclareMathOperator{\D}{d}
\DeclareMathOperator{\E}{e}

\begin{document}

%\topmargin -1.0in
%\textheight 10.5in
\pagestyle{empty}

\newcommand{\mline}{\vspace{.2in}\hrule\vspace{.2in}}


\title{\bf { AMATH 586 Spring 2022 \\ Homework 1 ---
Due April 8 on Gradescope by 11pm} }
\maketitle
\centerline{Be sure to do a {\tt git pull} to update your local version of the {\tt amath-586-2022} repository.}

\mline
\begin{enumerate}[label={\bf Problem~{\arabic*}:}]
  \item Using the Taylor series representation of the matrix exponential:
  \begin{enumerate}
\item Verify the identities
  \begin{align*}
    \frac{\D}{\D t} \E^{t A} = A \E^{t A} = \E^{t A} A
  \end{align*}
  for an $n\times n$ matrix $A$.
\item Verify that $u(t) = \E^{t A} \eta$ is indeed the solution of the IVP
  \begin{align*}
    \begin{cases}
      u'(t) = A u(t),\\
      u(0) = \eta.
    \end{cases}
  \end{align*}
\end{enumerate}
\mline
\item Construct a system (i.e., needs to be not scalar valued)
  \begin{align*}
    \begin{cases}
      u'(t) = f(u(t)),
    \end{cases}
  \end{align*}
  and two choices of initial data $u_0 \neq v_0$ so that two solutions
  \begin{align*}
    \begin{cases}
      u'(t) = f(u(t)),\\
      u(0) = u_0,
    \end{cases}
    \begin{cases}
      v'(t) = f(v(t)),\\
      v(0) = v_0,
    \end{cases}
  \end{align*}
  satisfy
  \begin{align}\label{1}
    \|u(t) - v(t)\|_2 = \|u(0) - v(0)\|_2 \E^{L t}
  \end{align}
  where $L$ a Lipschitz constant for $f(u)$.  Recall that we have shown that for any solution
  \begin{align*}
    \|u(t) - v(t)\|_2 \leq \|u(0) - v(0)\|_2 \E^{L t}.
  \end{align*}
  So, you are tasked with showing that this is sharp.  Then show that equality \eqref{1} fails to hold for $u'(t) = - f(u(t)),v'(t) = - f(v(t))$ with the same intial conditions.
  \mline
  \item Consider the IVP
\begin{align*}
\begin{cases}
u_1'(t) = 2u_1(t),\\
u_2'(t) = 3u_1(t) - u_2(t),
\end{cases}
\end{align*}
with initial conditions specified at time $t=0$.  Solve this problem in two
different ways:

\begin{enumerate}
\item Solve the first equation, which only involves $u_1$, and then insert
this function into the second equation to obtain a nonhomogeneous linear
equation for $u_2$.  Solve this using (5.8).  
Check that your solution satisfies the initial conditions and the ODE.

\item Write the system as $u' = Au$ and compute the matrix exponential using
(D.30) to obtain the solution.
\end{enumerate}

\mline

\item Consider the IVP
\begin{equation*}
\begin{cases}
u_1'(t) = 2u_1(t),\\
u_2'(t) = 3u_1 + 2u_2(t),
\end{cases}
\end{equation*}
with initial conditions specified at time $t=0$.  Solve this problem.
\mline

\item Consider the Lotka--Volterra system\footnote{This is a famous model of predator-prey dynamics.}
  \begin{align*}
    \begin{cases}
      u_1'(t) = \alpha u_1(t) - \beta u_1(t) u_2(t),\\
      u_2'(t) = \delta u_1(t) u_2(t) - \gamma u_2(t).
      \end{cases}
    \end{align*}
    For $\alpha = \delta = \gamma = \beta = 1$ and $u_1(0) = 5, u_2(0) = 0.8$ use the forward Euler method to approximate the solution with $k = 0.001$ for $t = 0,0.001,\ldots,50$.  Plot your approximate solution as a curve in the $(u_1,u_2)$-plane and plot your approximations of $u_1(t)$ and $u_2(t)$ on the same axes as a function of $t$.  Repeat this with backward Euler.  What do you notice about the behavior of the numerical solutions?  The most obvious feature is most apparent in the $(u_1,u_2)$-plane.

    \mline

    \item Determine the coefficients $\beta_0,~\beta_1,~\beta_2$ for the third
order, 2-step Adams-Moulton method.  Do this in two different ways:
\begin{enumerate} 
 \item Using the expression for the local truncation error in Section 5.9.1,
 \item Using the relation
 \[
 u(t_{n+2}) = u(t_{n+1}) + \int_{t_{n+1}}^{t_{n+2}}\,f(u(s))\,ds.
 \]
 Interpolate  a quadratic polynomial $p(t)$ through the three values
 $f(U^n),~f(U^{n+1})$ and $f(U^{n+2})$ and then integrate this polynomial
 exactly to obtain the formula.  The coefficients of the polynomial will
 depend on the three values $f(U^{n+j})$.   It's easiest to use the
 ``Newton form'' of the interpolating polynomial and consider the three
times $t_n=-k$, $t_{n+1}=0$, and $t_{n+2}=k$ so that $p(t)$ has the form
\[
p(t) = A + B(t+k) + C(t+k)t
\]
where $A,~B$, and $C$ are the appropriate divided differences based on the
data.  Then integrate from $0$ to $k$.   (The method has the same
coefficients at any time, so this is valid.)
\end{enumerate}
  \end{enumerate}

\end{document}

%%% Local Variables:
%%% mode: latex
%%% TeX-master: t
%%% End:
