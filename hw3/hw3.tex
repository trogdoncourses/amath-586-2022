\documentclass[10pt]{amsart}
\usepackage[margin=1.5in]{geometry}
\usepackage{amssymb,amsmath,enumitem}

\DeclareMathOperator{\D}{d}
\DeclareMathOperator{\E}{e}
\newcommand{\half}{\frac{1}{2}}
\newcommand\unp{U^{n+1}}
\newcommand\unm{U^{n-1}}
\newcommand{\bigo}{{\mathrm O}}
\newcommand{\reals}{\mathbb R}



\begin{document}

%\topmargin -1.0in
%\textheight 10.5in
\pagestyle{empty}

\newcommand{\mline}{\vspace{.2in}\hrule\vspace{.2in}}


\title{\bf { AMATH 586 Spring 2022 \\ Homework 3 ---
Due May 6 on GitHub by 11pm} }
\maketitle
\centerline{Be sure to do a {\tt git pull} to update your local version of the {\tt amath-586-2022} repository.}

\mline
\begin{enumerate}[label={\bf Problem~{\arabic*}:}]
\item It is natural to ask if a neighborhood of $z = 0$ can be in the absolute stability region $S$ for a LMM.  You will show that this cannot be the case.  Consider a consistent and zero-stable LMM
  \begin{align*}
    \sum_{j=0}^r \alpha_j U^{n+j} = k \sum_{j=0}^r \beta_j f(U^{n+j}).
  \end{align*}
  Recall the characteristic polynomial $\pi(\xi;z) = \rho(\xi) - z \sigma(\xi)$. Show:
  \begin{itemize}
  \item Consistency implies that $\pi(1;0) = 0$.
  \item Stability implies that $\rho'(1) \neq 0$.
  \item Suppose $\xi = 1 + \eta(z)$ for $z$ near zero so that $\pi(\xi;z) = \pi(1 + \eta(z);z) = 0$.  Compute $\eta'(0)$.  Why does this imply that there must be an interval $(0,\epsilon]$ for some small $\epsilon > 0$ that does not lie in the absolute stability region $S$.  
  \end{itemize}

  \mline
  
\item Consider the system of ODEs
  \begin{align*}
      \begin{bmatrix} u_1'(t) \\ u_2'(t) \\ u_3'(t) \\ u_4'(t) \end{bmatrix} = \begin{bmatrix} u_4(t) - \mu u_3(t) + \lambda g(u_1(t) + u_2(t))\\
        \mu u_3(t) - u_4(t) +  \lambda g(u_1(t) + u_2(t)\\
        - \sigma u_4(t) \\
      \sigma  u_3(t) \end{bmatrix}, \quad g(u) = u (1-u)^2, \quad \lambda > 0, ~ \sigma, \mu \in \mathbb R.
  \end{align*}
  You will show that for some choice of initial conditions the solution is bounded for all $t$.
  \begin{itemize}
  \item Solve for $u_3(t),u_4(t)$ in terms of $u_3(0) = \eta_3,u _4(0) = \eta_4$.
  \item Find an ODE solved by $w(t) := u_1(t) - u_2(t)$.  Solve it and show that for any fixed choice of $u_1(0) = \eta_1,u_2(0) = \eta_2,\eta_3,\eta_4$ that  $w(t)$ is bounded as $t \to \infty$.
  \item Find an ODE solved by $v(t) := u_1(t) + u_2(t)$.  If $0 < \eta_1 + \eta_2 < 1$ argue that $v(t)$ is bounded as $t \to \infty$.
  \item Why does this imply that $u_1(t), u_2(t)$ are bounded as $t \to \infty$?
  \end{itemize}

  \mline
\item Consider the system of ODEs from the previous problem, using
  \begin{align*}
    u_1(0) = 0.1, ~u_2(0) = 0.1, ~u_3(0) = 0, u_4(0) = 0.1.
  \end{align*}
  and $\lambda = 10$, $\sigma = 200$, $\mu = 100$, solve this problem with trapezoid with $k = 0.0001$.  We will use this as the ``ground truth'' for the solution. Plot the eigenvalues of the Jacobian as a function of $t$.  Now, solve with forward Euler and leapfrog (starting with one step of forward Euler) with $k = 0.001$.  You should find that one method goes unstable before the other.   Use the eigenvalues of the Jacobian to explain this instability. Explain why trapezoid is a good method for this problem.\\
  
\noindent
Note: If you sort the eigenvalues by their imaginary parts at each time step, things might be a bit clearer.

\mline

\item
  \begin{itemize}
    \item Plot the absolute stability region for the TR-BDF2 method (8.6).
  

    \item By analyzing $R(z)$, show 
that the method is both A-stable and L-stable.
Hint: To show A-stability, show that $|R(z)| \leq 1$ on the imaginary axis
and explain why this is enough.
\end{itemize}

\mline

\item Let $g(x)=0$ represent a system of $s$ nonlinear equations in $s$ unknowns,
so $x\in\reals^s$ and $g: \reals^s \to \reals^s$.  A vector $\bar
x\in\reals^s$ is a {\em fixed point} of $g(x)$ if 
\begin{equation}\label{a}
\bar x = g(\bar x).
\end{equation}
One way to attempt to compute $\bar x$ is with {\em fixed point iteration}:
from some starting guess $x^0$, compute
\begin{equation}\label{b}
x^{j+1} = g(x^j)
\end{equation}
for $j=0,~1,~\ldots$.

\begin{enumerate}
\item Show that if there exists a norm $\|\cdot\|$ such that $g(x)$ is
Lipschitz continuous with constant $L<1$ in a neighborhood of $\bar x$, then
fixed point iteration converges from any starting value in this
neighborhood.
{\bf Hint:} Subtract equation \eqref{a} from \eqref{b}.


\item Suppose $g(x)$ is differentiable and let $D_xg(x)$ be the $s\times s$
Jacobian matrix.  Show that if the condition of part (a) holds then
$\rho(D_xg(\bar x)) < 1$, where $\rho(A)$ denotes the spectral radius of a
matrix.

\item Consider a predictor-corrector method (see Section 5.9.4) consisting
of forward Euler as the predictor and backward Euler as the corrector, and
suppose we make $N$ correction iterations, i.e., we set
\begin{tabbing}
xxxxxxxxx\=xxxx\=\kill\\
\>$\hat U^0 = U^n + kf(U^n)$\\
\>for $j = 0,~1,~\ldots,~N-1$\\
\>\>$\hat U^{j+1} = U^n + kf(\hat U^j)$\\
\>\>end\\
\>$U^{n+1} = \hat U^N$.\\
\end{tabbing}
Note that this can be interpreted as a fixed point iteration for solving the
nonlinear equation
\[
\unp = U^n + kf(\unp)
\]
of the backward Euler method.  Since the backward Euler method is implicit
and has a stability region that includes the entire left half plane, as
shown in Figure 7.1(b), one might hope that this predictor-corrector method
also has a large stability region.

Plot the stability region $S_N$ of this method for $N=2,~5,~10,~20,~50$ 
and observe that in fact the stability region does not grow much in size.

\item Using the result of part (b), show that the fixed point iteration
being used in the predictor-corrector method of part (c) can only be
expected to converge if $|k\lambda| < 1$ for all eigenvalues $\lambda$ of
the
Jacobian matrix $f'(u)$.  

\item Based on the result of part (d) and the shape of the stability region
of Backward Euler, what do you expect the stability region $S_N$ of part (c)
to converge to as $N\to\infty$?

\end{enumerate}
\clearpage

\mline

\item Consider the matrix $M_r = I - r T$ where $T$ is the $m\times m$ matrix.
  \begin{align*}
    T = \begin{bmatrix}
      -2 & 1 \\
      1 & -2 & 1 \\
      & 1 &-2 & \ddots\\
      & &\ddots & \ddots & 1\\
      &&& 1 & -2 \end{bmatrix}
  \end{align*}
  and $r \geq 0$.  Find the largest value of $c$ such that $M_r$ is invertible for all $ r \in [0,c)$.

  
  \end{enumerate}

\end{document}

%%% Local Variables:
%%% mode: latex
%%% TeX-master: t
%%% End:
